\pagestyle{empty}

\noindent \textbf{\Large CD Használati útmutató}

\vskip 1cm

A CD gyökerében lévő szakdolgozat jegyzékben két jegyzék található. A dolgozat, és a program jegyzék. 

A dolgozat jegyzékben található a dolgozat LaTeX forráskódja, és a .pdf formátuma is, dolgozat.pdf néven. 

A program jegyzékben található a webalkalmazás forráskódja. 

Fel kell telepíteni a node.js-t, amit a https://nodejs.org/en/ oldalról lehet letölteni.

Node.js telepítése után, a parancssorban a következő utasítást kiadva: \texttt{npm install vue} feltelepül a Vue.js keretrendszer.


Miután elkészültünk ezekkel, a program jegyzékben találunk egy frontend és egy backend jegyzéket. Mind a két jegyzékben ki kell adni az alábbi parancsokat: 
\begin{itemize}
    \item \texttt{npm install}
    \item \texttt{npm run serve}
\end{itemize}


A frontend jegyzék parancssorában megjelenő ://Localhost címen fogjuk elérni a programot a böngészőnkből.
