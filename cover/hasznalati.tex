\pagestyle{empty}

\noindent \textbf{\Large CD Használati útmutató}

\vskip 1cm

\noindent A CD gyökerében lévő szakdolgozat jegyzékben két jegyzék található. A \texttt{dolgozat}, és a \texttt{program} jegyzék. 
\begin{itemize}
\item A \texttt{dolgozat} jegyzékben található a dolgozat \LaTeX\ forráskódja, és a dolgozat PDF formátumban \texttt{dolgozat.pdf} néven. 
\item A \texttt{program} jegyzékben található a webalkalmazás forráskódja.
\end{itemize}

\bigskip

\noindent Az alkalmazás indításához fel kell telepíteni a Node.js-t.\\
(Ez például a https://nodejs.org/en/ oldalról letöltve telepíthető.)

\bigskip

\noindent A Node.js telepítése után, a parancssorban a következő utasítást kiadva:
\begin{verbatim}
    npm install vue
\end{verbatim}
feltelepül a Vue.js keretrendszer.

Miután elkészültünk ezekkel, a program jegyzékben találunk egy \texttt{frontend} és egy \texttt{backend} jegyzéket. Mind a két jegyzékben ki kell adni az alábbi parancsokat: 
\begin{itemize}
    \item \texttt{npm install}
    \item \texttt{npm run serve}
\end{itemize}

A \texttt{frontend} jegyzék parancssorában megjelenő \texttt{//Localhost} címen fogjuk elérni a programot a böngészőnkből.
