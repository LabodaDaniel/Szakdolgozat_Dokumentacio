\Chapter{Technikai kivitelezés}
Ebben a fejezeetben bemutatom az elkészült programot a számomra legérdekesebb programrészletekkel. Nem térek ki minden funkcióra és programkódra, csak a lényegesebb egységekre. Részletezem a front end kivitelezését, amelybe tartozik a Vue JS, HTML és CSS kódok. Ezután a back end számításokat mutatom be, amelyet Node JS környezetben valósítottam meg.

\section{Front end kivitelezése}
Egy webalkalmazás front end része az, amit a felhasználó érzékel az oldalról. Itt részletesen bemutatom a főbb egységeket, melyek Vue JS-ben íródtak. A HTML és CSS részekre csak minimálisan térek ki, itt pedig igyekszem azokat a megoldásokat bemutatni, melyek a legizgalmasabbak voltak a kivitelezés során. 

\subsection{Autentikáció}
Az alkalmazás első oldala a bejelentkező/regisztrációs felület. A két oldal kissebb eltérések mellett ugyan az, ezért nem térek ki külön csak a változásokra. A register oldalon meg kell adnunk egy e-mail címet, egy hozzá tartozó jelszót, valamint egy becenevet is. Ha regisztráltunk, akkor a login oldalon csak megadjuk az e-mail címünket és a hozzá tartozó jelszót, a sign in gombra kattintunk és már be is enged minket az oldal. A bejelentkezést és a regisztrációt a Firebase végzi, valamint a validációt is. Az e-mail címnek tartalmaznia kell egy kukac jelet, illetve pontot és egy domain-t. A jelszónak legalább hat karakter hosszúnak kell lennie.

Ennek a két oldalnak a HTML része nagyon egyszerű. Egy formot belül két szöveges beviteli mezőt, ezen kívül egy gombot találunk. Mindezt közbezárja egy div, ami tartalmaz még két címet is, melyek a login és a sign up feliratok, ezekkel tudunk váltani a két oldal között. A login rész még kiegészül egy harmadik beviteli mezővel, ami a becenév.

\begin{python}
<div class="login">
  <h2 class="active"> Login </h2>
  <h2 class="nonactive"><router-link to="/register"> Sign up </router-link></h2> 
  <form @submit.prevent="Login">
    <input type="text" class="text" name="E-mail" v-model="email">
    <span>E-mail</span>
    <input type="password" class="text" name="password" v-model="password">
    <span>password</span>
    <button class="signin" value="Login">
      Sign In
    </button>
  </form>
</div>
\end{python}

Az autentikációt viszonylag egyszerű elvégezni a Firebase segítségével. Regisztrációnál a createUserWithEmailAndPassword függvényt használjuk, amivel létrehozunk egy felhasználót. Ehhez egy e-mail cím és egy jelszó tartozik, jelen esetben kieégüszlve egy becenévvel. Ezt el is tárolja nekünk a Firesotre Database-ben.Ezután bejelentkezésnél a signInWithEmailAndPassword függvényt használjuk, amely megkapja az e-mail és jelszó párost, és ha ez egyezik, már bent is vagyunk a főoldalon.

\begin{python}
const Register = () => {
      firebase
        .auth()
        .createUserWithEmailAndPassword(email.value, password.value)
        .then((user) => {
          db.collection("users")
            .doc(user.uid)
            .set({nickname: nickname.value})
        })
        .catch((err) => alert(err.message));
    };
\end{python}

Az autentikációs oldalak stílusa talán a leglátványosabb az alkalmazásban. Igyekeztem egységesen megformázni az összes oldalt. Ahol lehet lekerekítést használok, és mindenhol a azonos kék és piros színekkel kombinálok. A CSS legérdekesebb része számomra ebben a részben a két címsor formázása volt. A felhasználónak ez csupán színek váltakozása, viszont annál sokkal érdekesebb. Mikor melyik cím aktív, úgy annak a színe változik fehérré és kap egy aláhúzást, a másik címsor pedig elhalványodik. Ezt a login és a register oldal váltásával jelenítem meg.

\begin{python}
.active {
  border-bottom: 2px solid #1161ed;
}
.nonactive {
  color: rgba(255, 255, 255, 0.2);
}
\end{python}

\subsection{Főoldal és navigációs fejléc}
A főoldalról nem szeretnék hosszasan írni, mert tartalmilag is elég rövid. Mindössze két block elemben két paragrafus található benne, formázva. Az egyik röviden leírja a póker játék lényegét, a másik pedig megfogalmazza, miről is szól az alkalmazás.

A Vue JS egyik előnye, hogy komponensekből épül fel, melyeket többször fel tudunk használni. Nekem az egyik ilyen komponensem a navigációs fejléc. Ezt mindegyik oldalon felhasználom, hiszen a felhasználó ezen keresztül tud váltani az oldalak között. Másik sajátossága a view-router, ami megvalósítja az oldalak közötti ugrást, ezek az index.js file-ban vannak definiálva.

\begin{python}
<nav class="navbar">
  <ul>
    <li class="stand"><router-link to="/" class="must">Home</router-link></li>
    <li class="stand"><router-link to="/simulategame" class="must">Simulate Game</router-link></li>
    <li @click="Logout" class="log"><router-link to="/login" class="must">Logout</router-link></li>
  </ul>
</nav>
</template>
\end{python}

A Vue-nak köszönhetően ebben a komponensben nincs szükség javascriptre, anélkül tudunk váltani az oldalak között. Természetesen ezt is formázom CSS-el.

\subsection{Játék szimuláció}
Az alkalmazás nagy része front end részről a SimulateGame.vue oldalon van. Itt találjuk magát a játékot. Az oldalra érkezéskor a közepém egy pókerasztalt látunk, alatta pedig a 4 színt. Ezek közül egyre rákattintunk, akkor felugrik abból a színből mind a 13 lap, itt tudunk kiválasztani egyet. Ha ez megtörtént, akkor az a lap eltűnik a választási lehetőségek közül, többször már nem választhatjuk, valamint hozzáadja egy tömbhöz, majd ugyan ezt megismételjük 6-szor. Az összes lapot előre definiáltam. A lapokra való kattintáskor meghívódik a chooseCard() függvény, amely a fentieket végzi el.

\begin{python}
    chooseCard(name, id, url) {
      if (this.isThere === 8) {
        this.cardsFull = true;
        return;
      }
      this.sevenCards.push({ name, url });
      this.isThere = this.isThere + 1;
      this.showPic = false;
      this.isHidden = false;
      this.cards[id - 1] = false;
    }
\end{python}

A függvény elején rögtön egy feltétel van, ami ellenőrzi, hogy ne rakhassunk le 7 lapnál többet az asztalra. Amennyiben így szeretnénk tenni, egy pop up ablak ugrik fel, amely jelzi a felhasználónak, hogy több lapot már nem tehet le.

Ezen kívül a kiválasztott lapot belerakjuk a sevenCards globális objektumba, melyet majd továbbadunk a back end-nek, hogy végezze el a számításokat. Továbbá tartalmaz pár egyéb külső változót is, melyek a színek és lapok megjelenítéséért felelnek.

A másik lényeges rész ezen az oldalon az esélyeket megjelenítő diagram. Ez is egy külön komponens, viszont itt töltöm fel adatokkal. A back end-től megkapjuk az összes következő lapra számolt esélyt. Az ellenfélnek mindig 990-szer több esete lesz, hiszen nem ismerjük a lapjait. Emiatt a mi esélyeink első esetét össze kell hasonlítani az ellenfél eslő 990 esetével, ebből kapunk egy százalékos esélyt, amely majd az első oszlopunk lesz a diagrammon, és így tovább. 

Ezek mellett még olyan matematikai értékeket is számolok, mint az átlag, a medián vagy a szórás, melyek közül szerintem az utóbbi a legérdekesebb.

\begin{python}
        calculateDeviation(arr) {
      let mean =
        arr.reduce((acc, curr) => {
          return acc + curr;
        }, 0) / arr.length;
      arr = arr.map((k) => {
        return (k - mean) ** 2;
      });
      let sum = arr.reduce((acc, curr) => acc + curr, 0);
      return Math.sqrt(sum / arr.length);
    }
\end{python}

A felhasználó továbbá meg tud adni egy értéket is, amelyre az alkalmazás kiszámolja a relatív gyakoriságot, így többlet információhoz juthat, hogy mennyi esetben van például 50\% felett az ellenfél nyerési esélye.

\section{Back end kivitelezése}
A back end rész megvalósítása véleményem szerint a legkiemelkedőbb eleme a szakdolgozatomnak. Itt Node JS programkódok fognak szerepelni a hozzájuk tartozó magyarázattal, valamint bemutatom az adathalmazomat is.

\subsection{Adathalmaz bemutatása}
Úgy gondolom, átláthatóbb a dokumentáció úgy, ha először az adathalmazomat mutatom be, ugyanis erre többször fogok hivatkozni a későbbiekben. 

Mint azt az alapötlet szakaszban is említettem, először egy kb. 133 millió soros adathalmazzal kezdtem el a megvalósítást, melyben az összes lehetséges 7 lap kombinációja benne volt. Ezt pythonban magamnak generáltam le. A generáció futási ideje 7,5 óra volt, az adathalmaz mérete pedig 6 GB.

\begin{python}
from itertools import combinations

cards = combinations(["2c", "3c", "4c", ... , "As"], 7)
sevenCards = {}
sevenCards["cards"] = []

with open("data_new.json", "w") as outfile:
    for c in cards:
        c = list(c)
        outfile.write(f"{c}, \n")
\end{python}

Hamar kiderült, hogy ez akkora adathalmaz, amivel nagyon nehezen tudtam volna a továbbiakban dolgozni. Rengeteg keresést kell végeznem ezen, ami nagyon sok időbe telne, ezért más megoldást kellett találnom.

Az interneten találtam egy másik adathalmazt, amely lényegesen kisebb volt, 7462 soros, ezt használtam fel. 
\cite{chances}
Ez a lehetséges 5 lap kombinációinak száma úgy, hogy nem vizsgálunk külön minden színt. Így a számításaim annyival bővültek, hogy a kiválasztott 7 lapból \[ \binom{7}{5}=21\] keresést kell végeznem, hogy megtaláljam a legerősebb 5 lapot, amit a felhasználó magától is ki tud választani, viszont az alkalmazásnak is tudnia kell, tehát ezzel kell majd tovább számolnia.

Az adathalmazt JSON-ben tároltam el. Egyetlen objektumot tartalmaz, aminek az értéke a kombinációk, a kulcsa pedig a kéz erőssége. Azoknál az eseteknél, ahol számításba kell vennünk, hogy a lapok színe azonos, ott egy F betűt szúrtam az értékekbe. Ezeket abc sorrendbe rendeztem az egyszerűbb felhasználás végett.

\begin{python}
{
  "cardStrenght": {
    "AFJKQT": 1,
    "9FJKQT": 2,
    "89FJQT": 3,
    "789FJT": 4,
    "6789FT": 5,
    .
    .
    .
    "23457": 7462
    }
}
\end{python}

Az első 10 érték a színsorok, royal flush-el kezdve, a 11-dik az ász póker király kísérővel, a 7462-dik pedig a 7-es magaslap.