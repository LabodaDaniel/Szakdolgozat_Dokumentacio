\Chapter{Felhasznált technológiák}

\Section{Egyoldalas alkalmazások}

Ha figyelemmel követjük a Javascript keretrendszerek fejlődését, akkor láthatjuk, hogy egyre nagyobb teret kapnak az egyoldalas alkalmazások (Single Page Application), az olyan többoldalas alkalmazásokkal (Multi Page Application) szemben, mint a jQuery vagy a Laravel.

Három elterjedt Javascript keretrendszert különböztetünk meg. A Facebook által fejlesztett React-ot, az Angulart, amit a Google hozott létre és az Evan You alkotott Vue JS-t.

\Section{Vue JS bemutatása}

Mindhárom SPA-nak megvan a maga előnye és hátránya. Ezek közül inkább csak a Vue előnyeire koncentrálok a másik kettővel szemben. A három közül ez a legkönnyebben tanulható. Használatához nem szükséges különösebb meglévő Javascript tudás, enélkül is könnyen el lehet sajátítani.

A maga 18 KB-os méretével rendkívül kicsinek számít, amit könnyű letölteni és feltelepíteni, ennek ellenére pozitívan hat a felhasználói élményre és jó keresőmotor optimalizálással is rendelkezik. Saját virtuális DOM-ot (Document Object Model) renderel, ami jobban teljesít mint a React vagy az Angular. Könnyen olvasható a kódja, könnyen integrálható, jól dokumentált, és ezen kívül sok más előnye is van.

Többek között a Vue JS egyik fő tulajdonsága - ami megvan a többi SPA-ban is - hogy komponensekből épül fel, amiket újra felhasználhatunk, ezzel is elősegíti a rövidebb programkódot, átláthatóságot, egyszerűséget, amiket már fent említettem.

Ezek miatt esett a választásom a Vue JS-re, amire több forrás keretrendszerként, mások pedig könyvtárként hivatkoznak.

\Section{Node JS és az Express}

Amellett, hogy a legnépszerűbb programozási nyelv, a Javascript az egyik leginkább univerzális szoftverfejlesztési technológia. Hagyományosan frontend fejlesztésre használták, viszont az utóbbi időben elterjedt a szerver oldali (backend) használata is.Az egyik eszköz - és talán a legismertebb - ami ezt az elmozdulást elősegítette az a Node JS.

A Node JS tulajdonképpen nem egy keretrendszer, nem is egy könyvtár, hanem egy futtató környezet, ami a Chrome V8-as motorján alapszik. A technológiát először 2009-ben mutata be Ryan Dahl az Európai Javascript Konferencián. Amellett, hogy ezen a konferencián rögtön elnyerte a legizgalmasabb szoftver díjat, nem volt használatos széles körben. A technológia 2017-ben csúcsosodott ki, amikor is először használta egy ismertebb cég, a LinkedIn, vagy, hogy még egy párat említsek, a Netflix, eBay és az Uber.

Az egyik hatalmas előnye a Node JS használatának, hogy autómatikusan teljeskörű (full stack) web fejlesztővé válunk vele. Gondoljunk csak bele, két legyet ütünk egy csapásra. Egyszerűen nincs alternatívája a szerver oldali programozásnak a Javascript-ben, csak a Node JS, ez teszi megkerülhetetlenné a technológiát.

Ezek mellett gyors feldolgozású, mivel közös nyelvet használ a frontend és a backend, így a szinkronizáció gyors. Esemény alapú modell-t (event-based model) használ, ezért népszerű választás online játékok, vagy videó konferenciák készítéséhez. Gazdag az ökoszisztémája, az npm - ami a Node JS alapértelmezett csomagkezelője - paranccsal több, mint 800 ezer könyvtárat érhetünk el. Ezen felül több, számos előnye is van, amikre nem térek ki.

Express kifejtése.