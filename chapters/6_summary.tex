\Chapter{Összefoglalás}
Rengeteg tapasztalatot szereztem a szakdolgozatom megírása közben. Mivel még nem foglalkoztam webalkalmazás fejlesztéssel, így rengeteg tanulás előzte meg a program elkészítését. Megismerkedtem a Vue JS-el, a Node JS-el, és más nyelvekkel vagy keretrendszerekkel, amiket a szakdolgozatom készítése során felhasználtam. Szerencsére mindegyikük jól dokumentált környezet, így sok segítséget kaptam hozzájuk az interneten. Mivel napjainkban egyre inkább elterjedtek a webes alkalmazások, így remélem a jövőben is hasznomra válnak majd ezek az ismeretek.

Olyan témát sikerült választanom, amely az életben is közel áll hozzám, így nagy lelkesedéssel álltam hozzá az egész munkafolyamathoz. Megtaláltam a párhuzamot a póker és a matematika, statisztika között, amit véleményem szerint sikerült jól megvalósítani és bemutatni. A célomat, hogy egy "csaló programot" hozzak létre, csak megszorításokkal sikerült elérnem. Az alkalmazás tökéletesen megvalósítja az elképzeléseimet, kizárólag a futási idő akadályozó tényező az életben való felhasználásában. Hamar világossá vált számomra, hogy a fejlesztésnek egy nagyon fontos része a tervezés, hiszen az adja a program alapját. Érdekes volt számomra az is, hogy egy olyan alkalmazást sikerült létrehoznom, amelyhez hasonlót világhírű pókerrel foglalkozó társaságok alkottak meg.

A webalkalmazás gyakorlatban való felhasználásra a következők a megállapításaim. Mindenképpen tovább kellene optimalizálni a futási időt, hogy gördülékenyebben lehessen használni. Továbbá hasznos lehet reszponzívvá tenni az alkalmazást, hogy mobilról, táblagépről és más eszközökről is felhasználóbarát megjelenést kapjon. Ha pedig a felhasználók körét szeretnénk bővíteni, akkor az általam használt módszerrel más pókerfajták szimulációját is el lehetne készíteni. Ha csak a Texas Hold'Em-nél maradunk, akkor a fix limit és a pot limit változatát is meg lehetne valósítani, vagy akár az Omaha-t, 5 lapos pókert is.