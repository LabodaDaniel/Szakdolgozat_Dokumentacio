\Chapter{Bevezetés}

A póker a kártyajátékok közül minden felmérés szerint a dobogón végez elterjedtség és sok más szempontból is az egész világon. 2021-ben a póker volt a harmadik legnépszerűbb kártyajáték, azon belül is a legtöbbet játszott a Texas Hold’Em-nek a No Limit változata, vagy ahogy Dan Harrington pókervilágbajnok fogalmaz „a póker Cadillac-je”. 

Természetesen a pókerben is közrejátszik valamennyire a szerencse, így hivatalosan szerencsejátékként van bejegyezve. Ennek ellenére, ha egy jó játékos leül játszani egy kezdő játékossal, akkor mindig a jó játékos fog nyerni, többek között azért, mert a jó játékosok remekül meg tudják figyelni az asztalnál zajló eseményeket, melyeket logikusan tudnak felhasználni, hiszen a póker az információ játéka. Ezeket az információkat az arckifejezésekből, gesztikulálásból, a játékosok döntéseiből, emeléseinek méretéből és sok más egyéb öszszetevő mellett, a lapokból számolt matematikai esélyekből nyerik , amely az egyik legfontosabb részlete a játéknak, ha sikeresek szeretnénk lenni.

Szakdolgozatomban ezeket az esélyeket vizsgálom egy játékos szempontjából anélkül, hogy a saját lapjaimon, és az asztalon lévő lapokon kívül bármilyen információ rendelkezésemre állna, tehát csak a matematikai háttérrel foglalkozom. A profi játékosok általában fejben ki tudják számolni az adott esélyeket különböző leegyszerűsített módszerekkel, viszont ezek csak megközelítő eredményeket fognak adni, valamint csak egy számot kapnak, hogy hány százalék esélyük van. Szakdolgozatomban pontos esélyeket fogunk kapni, amelyeket nemcsak sémákra illesztve számolunk, hanem minden lehetőséget külön vizsgálunk. Ezen kívül nem csak egyetlen számot, hanem több információt is láttatni szeretnék, amely elősegíti a játékos döntését, hogy minél több esetben legyen nyereséges.

Mindezt webes környezetben valósítom meg, azon belül is a Vue JS keretrendszert használom, a stíluselemeket CSS-el hozom létre, a backend szervert pedig a Node JS szolgáltatja. Ezeken kívül más keretrendszereket, könyvtárakat és kisegítő lehetőségeket használok, melyeket részletezni fogok, továbbá bemutatom az alkalmazás létrejöttének lépéseit, az előforduló problémákat, ezekre a megoldásaimat és magát az alkalmazást. 