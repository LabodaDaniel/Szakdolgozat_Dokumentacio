\Chapter{Bevezetés}

A póker a világ egyik legnépszerűbb és legismertebb kártyajátéka. 2021-ben a póker volt a harmadik legnépszerűbb kártyajáték, azon belül is a legtöbbet játszott, a \linebreak Texas Hold’Em-nek a No Limit változata, vagy ahogy Dan Harrington pókervilágbajnok fogalmaz „a póker Cadillac-je” \cite{harrington}. 

A pókerben közrejátszik a szerencse, így hivatalosan szerencsejátékként jegyezték be. Ennek ellenére, ha egy tapasztalt játékos leül játszani egy kezdő játékossal, akkor több játszmából, hosszú távon átlagosan a tapasztalt játékos fog nyerni. A tapasztalt játékosok remekül meg tudják figyelni az asztalnál zajló eseményeket, melyeket logikusan tudnak felhasználni, hiszen a póker az információ játéka. Ezeket az információkat az arckifejezésekből, gesztikulációból, a játékosok döntéseiből, emeléseinek méretéből, és sok más egyéb összetevő mellett, a lapokból számolt matematikai esélyekből nyerik, amely az egyik legfontosabb részlete a játéknak, ha sikeresek szeretnénk lenni.

Szakdolgozatomban ezeket az esélyeket vizsgálom egy játékos szempontjából anélkül, hogy a saját lapjaimon, és az asztalon lévő lapokon kívül bármilyen információ rendelkezésemre állna, tehát csak a matematikai háttérrel foglalkozom. A profi játékosok általában fejben ki tudják számolni az adott esélyeket különböző leegyszerűsített módszerekkel, viszont ezek csak közelítő eredményeket fognak adni, valamint csak egy számot kapnak, hogy hány százalék esélyük van. Szakdolgozatomban bemutatom, hogy hogy kaphatunk pontos esélyeket, amelyeket nemcsak sémákra illesztve számolunk, hanem minden lehetőséget külön vizsgálunk. Ezen kívül nem csak egyetlen számot, hanem több információt is láttatni szeretnék, amely elősegíti a játékos megfelelő döntését, hogy minél több esetben legyen eredményes.

Az alkalmazást webes környezetben valósítom meg, azon belül is a Vue JS keretrendszert használom. A stíluselemeket CSS-el hozom létre, a backend oldalon lévő szerver alkalmazást pedig a Node JS szolgáltatja. Ezeken kívül más keretrendszereket, könyvtárakat és kisegítő lehetőségeket is használok, melyeket részletezni fogok, továbbá bemutatom az alkalmazás létrejöttének lépéseit, az előforduló problémákat, ezekre adott megoldásaimat és magát az alkalmazást.
